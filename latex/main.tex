\documentclass[
    11pt,
    aspectratio=1610
	]{beamer}
\setbeamertemplate{section page}{
  \begin{centering}
    \Large\insertsectionhead\par
  \end{centering}
}
\setbeamertemplate{footline}[frame number]

\usepackage[utf8]{inputenc}

\setbeamertemplate{caption}[numbered]
\usepackage{graphicx}
\usepackage{amsmath}
\hypersetup{
  colorlinks=true,    % make links colored
  linkcolor=blue,      % color for internal links (sections, TOC)
  urlcolor=blue,      % color for external hyperlinks
  citecolor=green     % color for citations
}

% Theme option to show section pages
% \setbeamertemplate{section page}[default]

% Automatically insert section page at each new section
\AtBeginSection{
  \begin{frame}[plain]
    \sectionpage
  \end{frame}
}


\title{Building a Multi-Asset Portfolio}
\subtitle{A case study on constructing a cryptocurrency portfolio}
\author{Harvey Huang}
\date{\today}

\begin{document}

\begin{frame}
  \titlepage
\end{frame}

%----------------------------+-----
\begin{frame}{Contents}
  \tableofcontents
\end{frame}

\section{Introduction}
\begin{frame}{Client profile}
  \begin{itemize}
    \item Age: 30
    \item Crypto-native
    % \item Annual net household income: \$100,000
    \item Investment capital: \$10,000
    \item Investment goal: maintain a balanced crypto portfolio
    \item Investment horizon: 2 years
    \item Preferred asset choice: cryptocurrencies only.
  \end{itemize}
\end{frame}

\begin{frame}{Advice question}
  \centering
  \it{What crypto asset and how much should the I invest?}
\end{frame}

\section{Methodology}
\begin{frame}{Proposed method: a quantitative analysis framework}
    \begin{itemize}
      \item Theoretical framework: Modern Portfolio Theory \cite{markowitz1952}
            \begin{itemize}
              \item Mean-variance analysis/optimization
              \item Balance return and risk (measured by standard deviation)
            \end{itemize}
      \item Steps:
        \begin{enumerate}
          \item Asset screening
          \item Data collection
          \item Preliminary analysis
          \item Portfolio optimization
        \end{enumerate}
        \item Performance assessment: simulation
  \end{itemize}
\end{frame}

\begin{frame}{Asset screening}
  \begin{itemize}
    \item Five cryptocurrenciesBalance between market cap (Nov. 2025) and 24h trading volume (\href{https://coinmarketcap.com/}{CoinMarketCap})
    \item Exclude meme coins (e.g., DOGE, SHIB), stablecoins (e.g., USDT, USDC).
    \item Mature infrastructure and ecosystem.
  \end{itemize}
  \ \\
  \ \\
  \begin{enumerate}
    \item Bitcoin (BTC-AUD)
    \item Ethereum (ETH-AUD)
    \item Ripple (XRP-AUD)
    \item Binance Coin (BNB-AUD)
    \item Solana (SOL-AUD)
\end{enumerate}
\end{frame}

\begin{frame}{Data collection}
  \begin{itemize}
    \item Yahoo Finance API.
    \begin{itemize}
      \item Source price data from CoinMarketCap (aggregation of multiple exchanges).
      \item Preferred since same asset may have different prices on different exchanges.
    \end{itemize} 
    \item Daily closing prices from Jan. 1, 2020 to Nov. 26, 2025.
    \begin{itemize}
      \item 24-7, weekends and holidays included.
      \item open = first trade post UTC 00:00; close = last trade pre UTC 00:00.
    \end{itemize}
  \end{itemize}
\end{frame}

\begin{frame}{Preliminary analysis on daily returns}
\begin{figure}[htbp] % h=here, t=top, b=bottom, p=page
    \centering
    \includegraphics[width=\textwidth]{figure/Figure_1.png}
    \caption{Daily return distribution}
    \label{fig:figure1}
\end{figure}
\end{frame}

\begin{frame}{Preliminary analysis on daily returns}
\begin{table}[]
\begin{tabular}{|c|c|c|c|c|c|}
\hline
Correlation& BNB-AUD & BTC-AUD & ETH-AUD & SOL-AUD & XRP-AUD \\ \hline
BNB-AUD & 1.00        & 0.60        & 0.63        & 0.50        & 0.45        \\ \hline
BTC-AUD & 0.60        & 1.00        & 0.78        & 0.51        & 0.52        \\ \hline
ETH-AUD & 0.63        & 0.78        & 1.00        & 0.59        & 0.55        \\ \hline
SOL-AUD & 0.50        & 0.51        & 0.59        & 1.00        & 0.43        \\ \hline
XRP-AUD & 0.45        & 0.52        & 0.55        & 0.43        & 1.00        \\ \hline
\end{tabular}
\end{table}
\end{frame}

\begin{frame}{Preliminary analysis on daily returns}
\begin{table}[]
\begin{tabular}{|c|c|c|c|c|c|}
\hline
 Daily & BNB-AUD & BTC-AUD & ETH-AUD & SOL-AUD & XRP-AUD \\ \hline
count    & 2052        & 2052        & 2052        & 2052        & 2052        \\ \hline
mean     & 0.003       & 0.002       & 0.002       & 0.004       & 0.003       \\ \hline
std. dev.      & 0.043       & 0.030       & 0.040       & 0.064       & 0.055       \\ \hline
skewness & 2.374       & -0.39       & 0.203       & 0.541       & 2.486       \\ \hline
kurtosis & 38.130      & 11.012       & 4.792       & 5.795       & 28.794      \\ \hline
min      & -0.327      & -0.150      & -0.266      & -0.416      & -0.425      \\ \hline
max      & 0.676       & 0.183       & 0.257       & 0.459       & 0.707       \\ \hline
\end{tabular}
\end{table}
\begin{itemize}
  \item V.S. all-in-BTC (annualized): expected return$=60.66\%$, std. dev.$=56.37\%$, Sharpe ratio$=1.076$
\end{itemize}
\end{frame}

\begin{frame}{Mean-variance optimization}
\[
\begin{aligned}
\max_{w_i} \frac{E[R_p] - R_f}{\sigma_p}\\
\text{s.t. }\sum_{i=1}^{n} w_i = 1, w_i \geq 0 \quad \forall i \\
\end{aligned}
\]

\begin{itemize}
  \item $E[R_p]$: expected portfolio return
  \item $R_f$: funding rate (assume 4\% p.a.)
  \item $\sigma_p$: portfolio standard deviation (risk)
  \item $w_i$: weight of asset $i$ in portfolio
\end{itemize}
\end{frame}

\begin{frame}{Mean-variance optimization}
\begin{figure}[htbp] % h=here, t=top, b=bottom, p=page
    \centering
    \includegraphics[width=\textwidth]{figure/Figure_2.png}
    \label{fig:figure2}
\end{figure}
\end{frame}

\begin{frame}{Mean-variance optimization}
\begin{itemize}
  \item Annualized expected return: 119\%
  \item Annualized volatility: 78\%
  \item Sharpe ratio: 1.51
\end{itemize}
\begin{figure}[htbp] % h=here, t=top, b=bottom, p=page
    \centering
    \includegraphics[width=0.5\textwidth]{figure/Figure_3.png}
    \label{fig:figure3}
\end{figure}
\end{frame}

\begin{frame}{Risk preference adjusted optimization}
\begin{itemize}
  \item Consider a model that factors in client's risk preferences \cite{Warren2019}.
  \item Power utility function (constant relative risk aversion):
\end{itemize}
\[
\max_{w_i} E[U(c)]
\]
\[
  U(c) = 
  \begin{cases}
  \frac{c^{1-\gamma} - 1}{1-\gamma}, & \gamma \neq 1, \\
  \ln(c), & \gamma = 2.
  \end{cases}
  \]
  \begin{itemize}
  \item $c$: consumption/wealth
  \item Small $\gamma$ (e.g., 1-3): risk-tolerant; large $\gamma$ (e.g., $\geq 5$): risk-averse
  \begin{itemize}
  \item Parameter calibration can be done via survey or experiments.
  \item For simplicity, assume $\gamma = 2$.
  \end{itemize}
\end{itemize}
\end{frame}

\begin{frame}{Risk preference adjusted optimization}
\begin{figure}[htbp] % h=here, t=top, b=bottom, p=page
    \centering
    \includegraphics[width=\textwidth]{figure/Figure_4.png}
    \label{fig:figure4}
\end{figure}
\end{frame}

\begin{frame}{Risk preference adjusted optimization}
\begin{itemize}
  \item Annualized expected return: 96\%
  \item Annualized volatility: 71\%
  \item Sharpe ratio: 1.35
\end{itemize}
\begin{figure}[htbp] % h=here, t=top, b=bottom, p=page
    \centering
    \includegraphics[width=0.5\textwidth]{figure/Figure_5.png}
    \label{fig:figure5}
\end{figure}
\end{frame}

\begin{frame}{Simulation}
\begin{itemize}
  \item Monte Carlo simulation of portfolio performance over 5 years.
  \begin{itemize}
    \item Euler discretization with daily time steps $(\Delta t=1/365$).
  \end{itemize}
  \item Asset modelling: Geometric Brownian Motion (GBM) with Heston stochastic volatility \cite{bergomi2015, heston1993}.
  \begin{itemize}
    \item Fat tails
    \item Volatility clustering
    \item Mean-reverting volatility
    \item Volatility co-movement.
  \end{itemize}
\end{itemize}
\end{frame}

\begin{frame}{Heston model}
For each asset $i = 1,\dots,5$, the price process satisfies
\begin{equation}
\label{eq:heston_price}
dS_{i,t}
= (r - q_i) S_{i,t} \, dt
+ \sqrt{v_{i,t}}\, S_{i,t} \, dW^{(S_i)}_t,
\end{equation}
where $r$ is the risk-free rate, $q_i$ is the continuous dividend (or 
foreign) yield ($0$ for cryptocurrencies), and $W^{(S_i)}_t$ is the Brownian motion driving the 
price of asset $i$.\\
\ \\
The instantaneous variance of each asset follows mean-reverting square-root diffusion:
\begin{equation}
\label{eq:heston_var}
dv_{i,t}
= \kappa_i (\theta_i - v_{i,t}) \, dt
+ \eta_i \sqrt{v_{i,t}} \, dW^{(v_i)}_t,
\qquad i = 1,\dots,5,
\end{equation}
\begin{equation}
 corr(dW^{(S_i)}_t, dW^{(v_i)}_t) = \rho_i
\end{equation}

where $\kappa_i > 0$ is the speed of mean reversion, 
$\theta_i > 0$ is the long-run variance level, and 
$\eta_i > 0$ controls the volatility of volatility.
\end{frame}

\begin{frame}{Model parameter calibration ($\kappa, \theta, \eta, \rho$) via moment matching}
\begin{table}[]
\begin{tabular}{|c|c|c|c|c|}
\hline
 & $\kappa$        &  $\theta$        & $\eta$       & $\rho$         \\ \hline
BTC-AUD   & 2.385836 & 0.000872 & 0.153398 & -0.135730 \\ \hline
ETH-AUD   & 2.189822 & 0.001606 & 0.218734 & -0.129252 \\ \hline
SOL-AUD   & 1.441285 & 0.004091 & 0.304743 & -0.205783 \\ \hline
BNB-AUD   & 1.580702 & 0.001827 & 0.482702 & -0.205482 \\ \hline
XRP-AUD   & 2.414101 & 0.002995 & 0.670663 & -0.325292 \\ \hline
\end{tabular}
\end{table}
\end{frame}

\begin{frame}{How good is the calibration?}
\begin{figure}[htbp] % h=here, t=top, b=bottom, p=page
    \centering
    \includegraphics[width=\textwidth]{figure/Figure_6.png}
    \label{fig:figure6}
\end{figure}
\end{frame}


\begin{frame}{How good is the calibration?}
\begin{table}[]
\begin{tabular}{|c|c|c|c|c|}
\hline
   & Actual Mean & Actual Std & Actual Skew & Actual Kurt \\ \hline
BTC-AUD & 0.002       & 0.029      & 0.170       & 6.635       \\ \hline
ETH-AUD & 0.002       & 0.040      & 0.202       & 7.783       \\ \hline
SOL-AUD & 0.004       & 0.064      & 0.540       & 8.778       \\ \hline
BNB-AUD & 0.003       & 0.043      & 2.370       & 41.033      \\ \hline
XRP-AUD & 0.003       & 0.055      & 2.485       & 31.750      \\ \hline
\end{tabular}
\end{table}

\begin{table}[]
\begin{tabular}{|c|l|l|l|l|}
\hline
   & Sim Mean  & Sim Std & Sim Skew  & Sim Kurt \\ \hline
BTC-AUD & –0.001175 & 0.047   & –0.626778 & 9.514    \\ \hline
ETH-AUD & –0.001721 & 0.066   & –0.640505 & 10.356   \\ \hline
SOL-AUD & –0.006384 & 0.105   & –1.135138 & 12.161   \\ \hline
BNB-AUD & –0.008859 & 0.113   & –1.641560 & 20.166   \\ \hline
XRP-AUD & –0.019456 & 0.142   & –2.332980 & 19.661   \\ \hline
\end{tabular}
\end{table}
\end{frame}

\begin{frame}{Sample simulation path: good}
\begin{table}[]
\begin{tabular}{|c|c|c|}
\hline
Annualized   & Sharpe optimal & Utility optimal \\ \hline
Return       & 185.67\%       & 161.00\%        \\ \hline
Std. dev.    & 94.35\%        & 71.64\%         \\ \hline
Max drawdown & -81.24\%       & -74.30\%        \\ \hline
\end{tabular}
\end{table}
\begin{figure}[htbp] % h=here, t=top, b=bottom, p=page
    \centering
    \includegraphics[width=0.7\textwidth]{figure/Figure_7.png}
    \label{fig:figure7}
\end{figure}
\end{frame}

\begin{frame}{Sample simulation path: worst}
\begin{table}[]
\begin{tabular}{|c|c|c|}
\hline
Annualized   & Sharpe optimal & Utility optimal \\ \hline
Return       & -298.94\%      & -93.00\%        \\ \hline
Std. dev.    & 112.15\%       & 77.60\%         \\ \hline
Max drawdown & -99.39\%       & -98.39\%        \\ \hline
\end{tabular}
\end{table}
\begin{figure}[htbp] % h=here, t=top, b=bottom, p=page
    \centering
    \includegraphics[width=0.7\textwidth]{figure/Figure_8.png}
    \label{fig:figure8}
\end{figure}
\end{frame}


\begin{frame}{Epilogue \& possible extensions}
  \begin{itemize}
    \item Consider a better model for long-term price dynamics.
    \item Out-of-sample backtesting and performance validation.
    \begin{itemize}
      \item Very tricky to work with time-series data due to effects like structural breaks, regime shifts, non-stationarity.
    \end{itemize}
    \item Dynamic asset allocation (moving $\%$ weights on assets).
    \begin{itemize}
      \item Rebalance portfolio periodically (e.g., monthly, quarterly).
      \item Model parameter recalibration with rolling window.
    \end{itemize}
    \item Model personal characteristics: tax, annual contribution/debt, expected consumption and withdrawal.
  \end{itemize}
\end{frame}


\begin{frame}{Q\&A}
  \centering
  Thank you! Any questions?\\
  Code \& results: \url{https://github.com/hsjharvey/temp_project}
\end{frame}



\section{References}
\begin{frame}{References}
  \begin{thebibliography}{10}
    \bibitem{bergomi2015}
      Bergomi, L. (2015). Stochastic volatility modeling. CRC press.
    \bibitem{Gemini2025}
      Google. (2025). Gemini. https://gemini.google.com/
    \bibitem{heston1993}
     Heston, S. L. (1993). A closed-form solution for options with stochastic volatility with applications to bond and currency options. The review of financial studies, 6(2), 327-343.
    \bibitem{copilot2025}
      Microsoft. (2025). Copilot. https://copilot.cloud.microsoft/
    \bibitem{markowitz1952}
      Markowitz, H. (1952). \textit{Portfolio Selection}. Journal of Finance, 7(1), 77-91.
    \bibitem{Warren2019}
      Warren, G. J. (2019). Choosing and using utility functions in forming portfolios. \textit{Financial Analysts Journal}, 75(3), 39-69.
  \end{thebibliography}
\end{frame}


\end{document}